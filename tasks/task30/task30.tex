\documentclass[a4paper, 10pt]{extarticle}

\usepackage{preamble}

\begin{document}
\title{Задание \textnumero 30\\ Структуры данных }
\author{}
\date{}
\maketitle

\section{Общая постановка задачи}

\begin{awesomeblock}[blue!70!black]{2pt}{\faTasks}{blue!70!black}
    В данном задании Вам предстоит познакомиться со стандартной библиотекой языка и с её помощью реализовать одну из структур данных.
\end{awesomeblock}

\begin{awesomeblock}[blue!70!black]{2pt}{\faTasks}{blue!70!black}
    Шаблоны задания могут содержать подключение некоторых дополнительных структур данных и функций, необходимых для одного из возможных решений. 
    
    Вы можете предложить другое решение, не использующее что-то из представленного. В таком случае необходимо удалить подключения неиспользуемого функционала.
\end{awesomeblock}

\begin{awesomeblock}[blue!70!black]{2pt}{\faTasks}{blue!70!black}
    В файле \mintinline{bat}"tests.rs", расположенном в папке \mintinline{bat}"test" находится минимальный набор тестов, которые помогут проверить частичную правильность решения на простом наборе входных данных.

    Для запуска тестов стоит воспользоваться командой \mintinline{bat}"cargo test".
\end{awesomeblock}

\begin{importantblock}
    Код программы должен быть корректно отформатирован. При наличии небрежностей форматирования преподаватель оставляет за собой право  не проверять решение и оценить его в 0 баллов. 

    Автоматически оформить решение можно с помощью команды \mintinline{bat}"cargo fmt".
\end{importantblock}

\begin{awesomeblock}[blue!70!black]{2pt}{\faFile}{blue!70!black}
    В качестве решения задания на портал должен быть загружено 1 файл "--- архив папки с проектом с именем \mintinline{bat}"task30-NN", где вместо \smlinline"NN" "--- номер вашего варианта.
    
    Архив должен включать в себя только исходные файлы и файлы проекта. Не следует включать в архив артефакты сборки (папку \verb|target|).
\end{awesomeblock}

\section{Ограничения}

\begin{importantblock}
    Объем вычислений должен минимально зависеть от размера входных данных и соответствовать требованиям, указанным в качестве комментария к функциям.

    В этом задании не накладывается ограничений на объем используемой памяти.
\end{importantblock}

\begin{importantblock}
    В этом задании не допускается использование стандартных реализаций структур, которые должны быть реализованы по заданию.
\end{importantblock}

\begin{importantblock}
Не следует делать предположений насчёт задания, не сформулированных явно в условии. Если возникают сомнения "--- задайте вопрос на форуме <<Язык Rust>>.
\end{importantblock}


%\newpage
\section{Пример задания}
% \noindent\textsc{Задание:}

\begin{tsk}[Пример 1. <<Глупый массив>>]
    Реализуем простой класс MyArray, который повторяет операции структуры Vec.
\end{tsk}

\section{Пример решения}
\begin{minted}[]{rust}
#![forbid(unsafe_code)]

struct MyArray<T> {
    data: Vec<T>,
}

impl<T> MyArray<T> {
    // Создание нового экземпляра MyArray<T>
    fn new() -> Self {
        MyArray { data: Vec::new() }
    }

    // Создание из класса Vec
    fn from(vec: Vec<T>) -> Self {
        MyArray { data: vec }
    }

    // Вставка нового элемента
    fn push(&mut self, value: T) {
        self.data.push(value);
    }

    // Получение элемента по заданному индексу
    fn get(&self, index: usize) -> Option<&T> {
        self.data.get(index)
    }

    // Получение длины массива
    fn len(&self) -> usize {
        self.data.len()
    }

    // Проверка на пустоту
    fn is_empty(&self) -> bool {
        self.data.is_empty()
    }
}  
\end{minted}

% Содержимое файла \verb"task27-00.scala":
% \inputminted[style= native]{scala}{task27-00.scala}

Текст примера (файл \verb"task30-00.rs") можно загрузить с портала.

\section{Задания}

\begin{tsk}[Бор]
Реализуйте структуру данных <<Бор>> (Trie). Структура данных хранит словарь и позволяет проверить, присутствует ли заданное слово в словаре.

Почитать об этой структуре данных можно \href{https://neerc.ifmo.ru/wiki/index.php?title=%D0%91%D0%BE%D1%80}{здесь}.
\end{tsk}

\begin{tsk}[Минимальный стек]
    Реализуйте структуру данных <<Минимальный стек>> (MinStack). Структура данных представляет собой стек, который дополнительно может быстро вернуть его минимальный элемент.

    Почитать об этой структуре можно \href{http://e-maxx.ru/algo/stacks_for_minima}{здесь}.
\end{tsk}

\begin{tsk}[Очередь с приоритетом/Минимальная куча]
    Реализуйте структуру данных <<Очередь с приоритетом>> по алгоритму минимальной кучи. Структура данных представляет собой очередь, которая позволяет быстро извлекать элементы по убыванию приоритета.

    Почитать об этой структуре данных можно \href{https://neerc.ifmo.ru/wiki/index.php?title=%D0%94%D0%B2%D0%BE%D0%B8%D1%87%D0%BD%D0%B0%D1%8F_%D0%BA%D1%83%D1%87%D0%B0}{здесь}.

    В задании не допускается использование стандартной структуры данных \verb|BinaryHeap|. Необходимо реализовать её функционал самостоятельно.
\end{tsk}


\begin{tsk}[Минимальная очередь. Бонус 25\%]
    Реализуйте структуру данных <<Минимальная очередь>> (MinQueue). Структура данных представляет собой очередь, которая дополнительно может быстро вернуть её минимальный элемент.

    Почитать об этой структуре можно \href{http://e-maxx.ru/algo/stacks_for_minima}{здесь}.
\end{tsk}

\begin{tsk}[Хэш"=таблица. Бонус 25\%]
    Реализуйте структуру данных <<Хэш-таблица>> по алгоритму открытого хэширования (\verb|HashMap|).

    Структура данных должна представлять собой массив определенной длины, в которой $i$-ый элемент "--- цепочка с ключами, хэш от которых равен $i$. Назовём эту цепочку <<ведром>> (bucket).

    Вставка элемента в структуру должна заключаться в получении соответствующего <<ведра>>, а затем поиска в нём элемента с вставляемым ключом. Если такого элемента не нашлось, то вставляем новую пару <<ключ"-значение>>, иначе обновляем значение по данному ключу.

    Удаление элемента заключается в удалении пары <<ключ"=значение>> из соответствующего <<ведра>>.


\end{tsk}

\begin{tsk}[LRU"=кэш. Бонус 25\%]
    Реализуйте кэш по алгоритму Least Recent Used (\verb|LRUCache|). 

    Структура данных должна позволять сохранять в ней одновременно не более \verb|capacity| элементов.

    Вставка элемента в структуру данных заключается в размещении пары <<ключ"=значение>>, если такой пары ранее не было или же обновления значения, если ключ уже присутствует.

    В случае, если число элементов равно \verb|capacity|, вставка вытесняет элемент, который не использовался дольше всех.
\end{tsk}

\end{document}
